\documentclass[12pt]{article}
\usepackage{geometry}
\geometry{margin=1in}
\usepackage{listings}
\usepackage{titlesec}
\titleformat{\section}{\large\bfseries}{}{0em}{}

\begin{document}

\begin{center}
    \LARGE \textbf{CS Assignment: Predicting Mango Types with Neural Nets} \\
    \vspace{0.5cm}
    \large Using Length, Mass, and Both as Features
\end{center}

\section*{Objective}
The goal of this assignment is to build intuition for neural networks by working with a simple mango dataset.  
Your dataset contains three columns:

\begin{itemize}
    \item \texttt{length} (cm)
    \item \texttt{mass} (grams)
    \item \texttt{type} (categorical label, e.g., mango variety)
\end{itemize}

You will design Python functions that attempt to predict mango type using:
\begin{enumerate}
    \item Length only
    \item Mass only
    \item Both length and mass
\end{enumerate}

---

\section*{File Structure}
Your project should follow this structure:
\begin{verbatim}
project/
  ├── data/
  │     └── mango_data.csv
  ├── src/
  │     ├── dataset_loader.py
  │     ├── mango_model.py
  │     └── train.py
  └── README.md
\end{verbatim}

---

\section*{Part 1: Dataset Loader}
In \texttt{dataset\_loader.py}, write a function to load the CSV.

\begin{lstlisting}[language=Python]
def load_dataset(path: str = "data/mango_data.csv"):
    # read csv
    # extract X (features) and y (labels)
    return X, y
\end{lstlisting}

---

\section*{Part 2: Prediction with Length Only}
In \texttt{mango\_model.py}, create a function to predict mango type using only length.  
It should apply a weighted sum, bias, and an activation function (e.g., sigmoid).

\begin{lstlisting}[language=Python]
def mango_predict_length(length, weight, bias):
    # weighted sum
    # apply activation
    # return prediction
\end{lstlisting}

---

\section*{Part 3: Prediction with Mass Only}
Repeat the process, but use only mass as the input.

\begin{lstlisting}[language=Python]
def mango_predict_mass(mass, weight, bias):
    # weighted sum
    # apply activation
    # return prediction
\end{lstlisting}

---

\section*{Part 4: Prediction with Length + Mass}
Extend to accept both features together.

\begin{lstlisting}[language=Python]
def mango_predict_two_features(length, mass, weight1, weight2, bias):
    # weighted sum with two inputs
    # apply activation
    # return prediction
\end{lstlisting}

---

\section*{Part 5: Loss Function}
Implement Mean Squared Error (MSE) in \texttt{mango\_model.py}.

\begin{lstlisting}[language=Python]
def compute_loss(y_true, y_pred):
    # implement MSE
    return loss
\end{lstlisting}

---

\section*{Part 6: Training Script}
In \texttt{train.py}, write a loop that:
\begin{enumerate}
    \item Initializes weights and bias
    \item Loops over the dataset
    \item Runs predictions
    \item Computes loss
    \item Adjusts weights using gradient descent
    \item Prints loss each epoch
\end{enumerate}

---

\section*{Deliverables}
\begin{itemize}
    \item \texttt{mango\_data.csv}
    \item Functions:
    \begin{itemize}
        \item \texttt{mango\_predict\_length()}
        \item \texttt{mango\_predict\_mass()}
        \item \texttt{mango\_predict\_two\_features()}
        \item \texttt{compute\_loss()}
    \end{itemize}
    \item \texttt{train.py} script that runs training
\end{itemize}

---

\section*{Stretch Goals}
\begin{itemize}
    \item Normalize your inputs (scale length/mass before using them)
    \item Add plots of the loss curve
    \item Implement a softmax classifier for multiple mango types
\end{itemize}

\end{document}
